%%%%%%%%%%%%%%%%%%%%%%%%%%%%%%%%%%%%%%%%%%%%%%%%%%%%%%%%%%%%
%% filename:	ks_help.tex
%% template:	Mon, 07 May 2012 13:01:14 +0200
%% author:	Hermann Lorenz
%% date:	19. Feb 2013 14:32
%%%%%%%%%%%%%%%%%%%%%%%%%%%%%%%%%%%%%%%%%%%%%%%%%%%%%%%%%%%%
\documentclass[%
	% für Package typearea
	paper=a4,% Druckbereich A4
	pagesize=auto	% Druckbereich als Papiergröße über-
			% nehmen
	]{scrartcl}

%%%%%%%%%%%%%%%%%%%%%%%%%%%%%%%%%%%%%%%%%%%%%%%%%%%%%%%%%%%%
%% Lokalisierung %%%%%%%%%%%%%%%%%%%%%%%%%%%%%%%%%%%%%%%%%%%
%%%%%%%%%%%%%%%%%%%%%%%%%%%%%%%%%%%%%%%%%%%%%%%%%%%%%%%%%%%%
\usepackage[utf8]{inputenc}	% Umlaute direkt eingeben
\usepackage[T1]{fontenc}	% Wörter mit Umlaute umbre-
				% chen
\usepackage[ngerman]{babel}	% deutsche Bezeichner
\usepackage[babel,german=quotes]{csquotes}	% \enquote{}

%%%%%%%%%%%%%%%%%%%%%%%%%%%%%%%%%%%%%%%%%%%%%%%%%%%%%%%%%%%%
%% Tabellen %%%%%%%%%%%%%%%%%%%%%%%%%%%%%%%%%%%%%%%%%%%%%%%%
%%%%%%%%%%%%%%%%%%%%%%%%%%%%%%%%%%%%%%%%%%%%%%%%%%%%%%%%%%%%
\usepackage{tabularx}
\usepackage{booktabs}	% \toprule\midrule\bottomrule
			% \addlinespace

%%%%%%%%%%%%%%%%%%%%%%%%%%%%%%%%%%%%%%%%%%%%%%%%%%%%%%%%%%%%
%% Bilder %%%%%%%%%%%%%%%%%%%%%%%%%%%%%%%%%%%%%%%%%%%%%%%%%%
%%%%%%%%%%%%%%%%%%%%%%%%%%%%%%%%%%%%%%%%%%%%%%%%%%%%%%%%%%%%
\usepackage{graphicx}	% \includegraphics{bild.pdf}

%%%%%%%%%%%%%%%%%%%%%%%%%%%%%%%%%%%%%%%%%%%%%%%%%%%%%%%%%%%%
%% Mathematische Symbole %%%%%%%%%%%%%%%%%%%%%%%%%%%%%%%%%%%
%%%%%%%%%%%%%%%%%%%%%%%%%%%%%%%%%%%%%%%%%%%%%%%%%%%%%%%%%%%%
\usepackage{amssymb}
\usepackage{amsmath}
\usepackage{amsfonts}

%%%%%%%%%%%%%%%%%%%%%%%%%%%%%%%%%%%%%%%%%%%%%%%%%%%%%%%%%%%%
%% Sonstige Symbole %%%%%%%%%%%%%%%%%%%%%%%%%%%%%%%%%%%%%%%%
%%%%%%%%%%%%%%%%%%%%%%%%%%%%%%%%%%%%%%%%%%%%%%%%%%%%%%%%%%%%
\usepackage{eurosym}
\usepackage[italian,	% \unita, conflict with babel
	squaren		% \squaren, conflict with amssymb
	]{SIunits}

\usepackage{listings}
\lstset{basicstyle=\ttfamily}

%%%%%%%%%%%%%%%%%%%%%%%%%%%%%%%%%%%%%%%%%%%%%%%%%%%%%%%%%%%%
%% pdf-links %%%%%%%%%%%%%%%%%%%%%%%%%%%%%%%%%%%%%%%%%%%%%%%
%%%%%%%%%%%%%%%%%%%%%%%%%%%%%%%%%%%%%%%%%%%%%%%%%%%%%%%%%%%%
\usepackage[ngerman]{varioref}	% \vpageref{}
\usepackage[ngerman,	% deutsche Bezeichnungen
	pdftex		% in Links Umbrüche erlauben
	]{hyperref}
\hypersetup{
	pdfauthor={Hermann Lorenz},
	pdftitle={NOCH KEIN TITEL}
	}	% \autoref{}

%%%%%%%%%%%%%%%%%%%%%%%%%%%%%%%%%%%%%%%%%%%%%%%%%%%%%%%%%%%%
%% eigene Macros %%%%%%%%%%%%%%%%%%%%%%%%%%%%%%%%%%%%%%%%%%%
%%%%%%%%%%%%%%%%%%%%%%%%%%%%%%%%%%%%%%%%%%%%%%%%%%%%%%%%%%%%
\newcommand{\vautoref}[1]{\autoref{#1}\vpageref{#1}}

\begin{document}
\section{Laplace-Transformation}
\minisec{Idee}
	Überführung einer Gleichung in den Laplace-Raum, in diesem
	sind die einzelnen Operationen einfacher durchzuführen.
	Wenn das Ergebnis gefunden ist, überführen in den normalen Raum.

	\(L\left[f(x)\right](s)\) \dots Laplacetransformation auf die
	Funktion \(f(x)\) anwenden und  Variable \(s\) einführen.
\minisec{Beispiel}
\begin{align*}
	a \dot y(t) + a_0 y(t) & = b(t)\\
	L\left[a_1 \dot y(t) + a_0 y(t)\right]
		& = L\left[b(t)\right]\\
	L\left[a_1 \dot y(t) + a_0 y(t)\right](s)
		& = L\left[b(t)\right](s)
		& L\left[a+b\right] = L\left[a\right] + L\left[b\right]\\
	L\left[a_1 \dot y(t)\right](s) + L\left[a_0 y(t)\right](s)
		& = L\left[b(t)\right](s)
		& L\left[c \cdot f(x)\right] = c \cdot L\left[f(x)\right]\\
	a_1  \cdot L\left[\dot y(t)\right](s) + a_0  \cdot L\left[y(t)\right](s)
		& = L\left[b(t)\right](s)
		& L\left[ f'(t) \right](s) = sF(s) - f(0)\\
	a_1 \left[s Y(s) - y(0)\right] + a_0 y(s)
		& = L\left[b(t)\right](s)
\end{align*}

\minisec{Praktikum 9, Aufgabe 1}
\begin{align*}
	4 \cdot y^{(4)}(t) - 12 y'''(t) + 11 y''(t) - 3 y'(t)
		& = 4 cos(t)\\
	L\left[4 \cdot y^{(4)}(t) - 12 y'''(t) + 11 y''(t) - 3 y'(t)\right](s)
		& = L\left[4 cos(t)\right](s)\\
	4 \cdot L\left[y^{(4)}(t)\right](s)
		- 12 L\left[y'''(t)\right](s)
		+ 11 L\left[y''(t)\right](s)
		- 3 L\left[y'(t)\right](s)
		& = 4 L\left[\cos(t)\right](s)
\end{align*}
Formeln nach \url{http://de.wikipedia.org/wiki/Laplacetransformation#Allgemeine_Eigenschaften} transformieren:
\begin{align*}
	f'(t)	& = s F(s) - f(0)\\
	f''(t)	& = s^2 F(s) - sf(0) - f'(0)\\
	f^{(n)}t)
		& = s^n F(s) - \sum\limits_{k=0}^{n-1} f^{(k)}(0) s^{n-k-1}\\
	f'''(t)	& = s^3 F(s) - s^2 f(0) - sf'(0) - f''(0)\\
	f^{(4)}(t)	& = s^4 F(s) -s^3 f(0) - s^2 f'(0) - sf''(0) - f'''(0)\\
	%\cos(at) \cdot f(t)	& = \frac{1}{2} \cdot (F(s - ia) + F(s + ia))
	\cos(at) & = \frac{s}{s^2+a^2}
\end{align*}
In obige Formel einsetzen:
\begin{align*}
	4 \cdot \underbrace{\left(s^4 F(s) -s^3 f(0) - s^2 f'(0) - sf''(0) - f'''(0)\right)}_{L\left[y^{(4)}(t)\right](s)}\\
		- 12 \underbrace{\left(s^3 F(s) - s^2 f(0) - sf'(0) - f''(0)\right)}_{L\left[y'''(t)\right](s)}\\
		+ 11 \underbrace{\left(s^2 F(s) - sf(0) - f'(0)\right)}_{L\left[y''(t)\right](s)}
		- 3 \underbrace{\left(s F(s) - f(0)\right)}_{L\left[y'(t)\right](s)}
		& = 4 \underbrace{\left(\frac{s}{s^2 + 1}\right)}_{L\left[\cos(t)\right](s)}
\end{align*}
	
	\[F(s) = L\left[y(t)\right](s)\]

	Automatisch in Matlab mit \lstinline=laplace(cos(t), t, s)= transformieren.

	\begin{lstlisting}[gobble=16]
		syms f(t)
		Df = diff(f(t), t);
		SS = laplace(Df, t, s);
	\end{lstlisting}

	Jetzt \(X(s)\), damit meint er hier \(F(s)\) bzw. \(Y(s)\), ausklammern:
\begin{align*}
	4 \cdot \left(s^4 F(s) -s^3 f(0) - s^2 f'(0) - sf''(0) - f'''(0)\right)\\
		- 12 \left(s^3 F(s) - s^2 f(0) - sf'(0) - f''(0)\right)\\
		+ 11 \left(s^2 F(s) - sf(0) - f'(0)\right)\\
		- 3 \left(s F(s) - f(0)\right)
		& = 4 \left(\frac{s}{s^2 + 1}\right)\\
	4 s^4 F(s) - 4 s^3 f(0) - 4 s^2 f'(0) - 4 sf''(0) - 4 f'''(0)\\
		- 12 s^3 F(s) + 12 s^2 f(0) + 12 sf'(0) + 12 f''(0)\\
		+ 11 s^2 F(s) - 11 s f(0) - 11 f'(0)\\
		- 3 s F(s) + 3 f(0)
		& = 4 \frac{s}{s^2 + 1}\\
	F(s) \left(4 s^4 - 12 s^3 + 11 s^2 - 3 s\right)
		+ f(0) \left(- 4 s^3 + 12 s^2 - 11 s + 3\right)\\
		+ f'(0) \left(- 4 s^2 + 12 s - 11\right)
		+ f''(0) \left(12 - 4s\right)
		- 4 f'''(0)
		& = 4 \frac{s}{s^2 + 1}\\
\end{align*}

Zwischendurch die Anfangsbedingungen
\begin{align*}
	y(0) = f(0) &= 5\\
	y'(0) = f'(0) &= 4\\
	y''(0) = f''(0) &= 3\\
	y'''(0) = f'''(0) &= 2
\end{align*}
einsetzen.
\begin{align*}
	F(s) \left(4 s^4 - 12 s^3 + 11 s^2 - 3 s\right)
		+ 5 \left(- 4 s^3 + 12 s^2 - 11 s + 3\right)\\
		+ 4 \left(- 4 s^2 + 12 s - 11\right)
		+ 3 \left(12 - 4s\right)
		- 4 \cdot 2
		& = 4 \frac{s}{s^2 + 1}\\
	F(s) \left(4 s^4 - 12 s^3 + 11 s^2 - 3 s\right)
		- 20 s^3 + 60 s^2 - 55 s + 15\\
		- 16 s^2 + 48 s - 44
		+ 36 - 12s
		- 8
		& = 4 \frac{s}{s^2 + 1}\\
	F(s) \left(4 s^4 - 12 s^3 + 11 s^2 - 3 s\right)
		- 20 s^3
		+ 44 s^2
		- 19 s
		- 1
		& = 4 \frac{s}{s^2 + 1}\\
	F(s)
		& = \frac{4 \frac{s}{s^2 + 1}
		+ 20 s^3
		- 44 s^2
		+ 19 s
		+ 1
		}{
			4 s^4 - 12 s^3 + 11 s^2 - 3 s
		}
		\\
\end{align*}

\appendix
\section{MATLAB}
	\lstinline=clc=
	Anzeige löschen

	\lstinline=clear all=
	alle Variablen löschen

	\lstinline=clear x=
	Variable \(x\) löschen

	\lstinline=close all=
	alle Fenster (z.\,B. Plots) schließen

	\lstinline=syms f(t) s=
	Symbol ohne Wert anlegen

	\lstinline=laplace(cos(t), t, s)=
	Laplacetransformation durchführen

	\lstinline=ilaplace(Formel, s, t)=
	Laplace\emph{rück}transformation durchführen

	\lstinline=diff(cos(t), t)=
	Ableitung von \(\cos(t)\) nach \(t\)




\section{Simulink}
	Mit \emph{Integratoren} die verschiedenen Ableitungen darstellen.
	Die Gleichung nach \(f^{(n)}(t) = \dots\) umformen und aufbauen.

	Das Simulationsprogramm kann mit \lstinline=sim('simname')= ausgeführt
	werden.

\subsection{Continuous}
	\begin{description}
	\item[Integrator]
		Ableitung erstellen \emph{mit Anfangsbedingungen} in
		den Eigenschaften
	\end{description}

\subsection{Math Operations}
	\begin{description}
	\item[Gain]
		mit einem festen Wert multiplizieren
	\item[Trigonometric Function]
		für \(\sin\)- oder \(\cos\)-Funktionen
	\end{description}

\subsection{Sinks}
	\begin{description}
	\item[Display]
		konkreten Wert ausgeben
	\item[Scope]
		Diagramm ausgeben
	\end{description}

\subsection{Sources}
	\begin{description}
	\item[Constant]
		Konstanter Wert, z.\,B. für Addition oder Multiplikation
	\item[Clock]
		entspricht der Eingangsvariablen \(t\)
	\end{description}
\end{document}
